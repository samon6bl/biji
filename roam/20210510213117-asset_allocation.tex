% Created 2021-09-07 二 19:06
% Intended LaTeX compiler: pdflatex

\documentclass[12pt,a4paper]{article}
\usepackage{nopageno}
\usepackage{hyperref}
\usepackage{fontspec}
\usepackage{etoolbox}
\usepackage[margin=2cm]{geometry}
\usepackage[dvipdfmx]{graphicx}
\usepackage{longtable}
\usepackage{float}
\usepackage{wrapfig}
\usepackage{rotating}
\usepackage[normalem]{ulem}
\usepackage{amsmath}
\usepackage{textcomp}
\usepackage{marvosym}
\usepackage{wasysym}
\usepackage{multicol}
\usepackage{amssymb}
\usepackage{fancyhdr}
\usepackage[cache=false]{minted}
\tolerance=1000
\setsansfont{Source Han Sans SC}
\setromanfont{Source Han Serif SC}
\setmonofont[Scale=0.9]{Inziu Iosevka Slab SC}
\newfontfamily\quotefont{Source Han Serif SC}
\newfontfamily\headfootfont{Source Han Sans SC}
\AtBeginEnvironment{quote}{\quotefont\small}
\XeTeXlinebreaklocale ``zh''
\XeTeXlinebreakskip = 0pt plus 1pt
\linespread{1.0}
\hypersetup{
  colorlinks=true,
  linkcolor=[rgb]{0,0.37,0.53},
  citecolor=[rgb]{0,0.47,0.68},
  filecolor=[rgb]{0,0.37,0.53},
  urlcolor=[rgb]{0,0.37,0.53},
  pagebackref=true,
  linktoc=all,}
\renewcommand{\headrulewidth}{0.4pt}
\renewcommand{\footrulewidth}{0.4pt}
\pagestyle{fancy}
\fancyfoot[C]{} % Clear page number
\fancyhead[RE]{\headfootfont\small\leftmark} % 在偶数页的右侧显示章名
\fancyhead[LO]{\headfootfont\small\rightmark} % 在奇数页的左侧显示小节名
\fancyhead[LE,RO]{\headfootfont\small~\thepage~} % 在偶数页的左侧,奇数页的右侧显示页码
\usepackage{fontspec}
\setmainfont{Noto Serif CJK SC}
\author{Bolun Liu}
\date{\today}
\title{Asset Allocation}
\hypersetup{
 pdfauthor={Bolun Liu},
 pdftitle={Asset Allocation},
 pdfkeywords={},
 pdfsubject={},
 colorlinks=true,
 linkcolor=black
}
\begin{document}

\maketitle
\tableofcontents

\textbf{上午主观题} 中比较容易拿分的科目
\section{Overview of Asset Allocations}
\label{sec:orgb323ae1}
\subsection{Investment Governance(投资规划)}
\label{sec:org6759c0b}
achieve the asset owner's \texttt{investment objectives} within the asset owner's \texttt{tisk tolerance and constraints}.
\begin{enumerate}
\item Articulating investment objectives
\begin{itemize}
\item Defined benefit pension fund
\item Endowment fund
\item individual investor
\end{itemize}
\item Allocation of rights and responsibilities
\item investment policy statement
\item Asset allocation and rebalancing policy
\item Reporting framework
\item The governance audit
\end{enumerate}
\subsection{A Quick Glance of Asset Allocation}
\label{sec:org4d40019}
\subsubsection{\textbf{Stratefic asset allocation(SAA)} \texttt{长期}}
\label{sec:orge11a20b}
SAA is an asset allocation that arises in \texttt{long-term} investment planning
\begin{itemize}
\item 9 steps of SAA
\begin{itemize}
\item return objectives
\item risk tolerance
\item investment horizon
\item constraints
\item approach to asset allocation
\item \href{20210416100520-capital_market_expectations.org}{Capital Market Expectations}
\item asset allocation choices for consideration
\item Test the robustness of the potential choices
\item back to step
\end{itemize}
\end{itemize}
\subsubsection{Tactical Asset Allocation(TAA)(战术) \texttt{短期}}
\label{sec:org6a6f4da}
\begin{itemize}
\item \textbf{short-term deviations} from strategic asset allocation
\item TAA is \textbf{active management} at the asset class level
\end{itemize}
\subsubsection{Economic Balance Sheet}
\label{sec:org7cb8bd5}
accounting balance sheet + \textbf{extended protfolio assets and liabilities}
\subsection{Asset Allocation Approaches}
\label{sec:org2276592}
\subsubsection{Three Categories}
\label{sec:orgea39829}
\texttt{investment problem}
\begin{enumerate}
\item Asset-only approach
\label{sec:orgce3479f}
maximize \texttt{Sharpe raitio}
eg. \uline{mean-variance optimiztion(MVO)}
\begin{itemize}
\item Risk measures
\begin{itemize}
\item Volatiliity(standard deviation)
\item Tracking risk(tracking error, risk relative to benchmark)
\item Downside risk(semi-variance, maximum drawdown, VaR)
\end{itemize}
\end{itemize}
\item Liability-relative approach
\label{sec:org3bc14a5}
objective of \texttt{funding liabilities}
eg. \textbf{surplus(A-L) optimization}
\begin{itemize}
\item risk measures:
\begin{itemize}
\item Volatility of surplus(A-L)
\item Volatility of fund(A/L)
\end{itemize}
\end{itemize}
\item Goals-based approach
\label{sec:orga75cde9}
\texttt{sub-portfolios} for each specified goals
risk limits can be quatified as the \texttt{maximum acceptable probability} of not achieving a goal
\end{enumerate}
\subsubsection{Two Categories}
\label{sec:orge5a4c8a}
\texttt{exposures} 具体、抽象
\textbf{*} Asset class-based approach
each asset class has its own \texttt{systematic risk exposure}.
\begin{itemize}
\item Five criteria to effectively specify asset class for the purpose of asset allocation:
\item Assets within an asset class should be relatively \texttt{homogeneous}
\item Asset classes should be \texttt{diversifying}.
\item Asset classes should be \texttt{mutually exclusive}.
\item The asset classes as a group should make up a preponderance of \texttt{world investable wealth}.
\item Asset classes selected for investment should have the capacity to absorb a meaningful prportion of \texttt{an investor's portfolio}.
\end{itemize}
\subsubsection{Factor-based approach}
\label{sec:orgcf5d4fa}
desired exposures to \texttt{specified risk factors}.
\uline{Multifactor risk models}
\begin{itemize}
\item Steps
\begin{itemize}
\item Specify risk factors and the desired exposure to each risk factor
\item Construct \texttt{factor portfolios} that isolate exposure to each risk factor.
\item Map the exposure in \textbf{factor space} back to \textbf{asset class space}.
\end{itemize}
\end{itemize}
\texttt{zero dollar investment}
\subsection{Implementation of Asset Allocation}
\label{sec:orgf552c2f}
\uline{passive/active choice}
\begin{itemize}
\item passive : follow benchmark(Index, liability)
\item active : beat benchmark

\item Factors on Passive/Active Choices
\begin{itemize}
\item Available investments
\item Scalability of active strategies
\item Trade-off benefits and risks
\item Feasibility of investing passively while incorporating client-specific constraints
\item Beliefs concerning market informational efficiency
\item Tax status
\end{itemize}
\end{itemize}
\subsection{Reblancing Asset Allocation}
\label{sec:org56a4bd5}
\textbf{Rebalancing} is the discipline of adjusting portfolio weights to mre closely align with \texttt{SAA}.
\subsubsection{Approaches to Rebalancing}
\label{sec:org61f9534}
\begin{enumerate}
\item Calendar rebalancing
\label{sec:org777d93e}
\texttt{on a periodic basis}
eg. monthly, quarterly, semiannually
\item Percent-range rebalancing
\label{sec:orgc6727c2}
\end{enumerate}
\section{Principles of Asset Allocation}
\label{sec:orgea8c978}
\subsection{Asset-Only Allocation}
\label{sec:orgd089649}
\subsubsection{Mean-Variance Optimiztion}
\label{sec:org1dfc579}
\begin{enumerate}
\item Introduction of MVO
\label{sec:org912d7aa}
\texttt{efficient portfolio} use MVO to determine
inputs:returns,risks(Std.Dev.),and pair-wise correlations
\begin{enumerate}
\item Utility Function
\label{sec:orgcd3b9c0}
\(U_{m}=E(R_{m})-0.005\lambda \sigma^{2}_{m}\)
\(\lambda=1\dots10\)
\item Safety-first ratio
\label{sec:org1b3e102}
\(SFR=\frac{r_{p}-MAR}{\sigma^{2}}\)
\begin{quote}
Typically, this single set weights is \texttt{relatively extreme} with very large long and short positions in each asset class
\end{quote}
\item Two constraints
\label{sec:org3a01eba}
\begin{enumerate}
\item \textbf{Budget constraint}: weights must sum to 1.
\item \textbf{Non-negativity constraint}: allow only positive weights.
\end{enumerate}
\end{enumerate}
\item Criticisms of MVO
\label{sec:org9f59e8e}
\begin{enumerate}
\item The asset allocations are highly sensitive to the inputs, especial the expected return(garbage in,garbage out)
\end{enumerate}
\item Other Issues of MVO
\label{sec:org72be8ff}
\end{enumerate}
\subsubsection{Other Models}
\label{sec:org2567323}
\subsection{Liability-Relative Allocation}
\label{sec:org071c559}
\subsubsection{Introduction of Liability-Relative Allocation}
\label{sec:orga0853e9}
\subsubsection{Approaches of liability-Relative Allocation}
\label{sec:org5855434}
\subsection{Goals-Based Allocation}
\label{sec:org353e0a4}
\subsection{Other Allocation Approaches}
\label{sec:orge759c77}
\end{document}
