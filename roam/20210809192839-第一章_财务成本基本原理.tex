% Created 2021-08-11 三 15:24
% Intended LaTeX compiler: pdflatex

\documentclass[12pt,a4paper]{article}
\usepackage{nopageno}
\usepackage{hyperref}
\usepackage{fontspec}
\usepackage{etoolbox}
\usepackage[margin=2cm]{geometry}
\usepackage[dvipdfmx]{graphicx}
\usepackage{longtable}
\usepackage{float}
\usepackage{wrapfig}
\usepackage{rotating}
\usepackage[normalem]{ulem}
\usepackage{amsmath}
\usepackage{textcomp}
\usepackage{marvosym}
\usepackage{wasysym}
\usepackage{multicol}
\usepackage{amssymb}
\usepackage{fancyhdr}
\usepackage[cache=false]{minted}
\tolerance=1000
\setsansfont{Source Han Sans SC}
\setromanfont{Source Han Serif SC}
\setmonofont[Scale=0.9]{Inziu Iosevka Slab SC}
\newfontfamily\quotefont{Source Han Serif SC}
\newfontfamily\headfootfont{Source Han Sans SC}
\AtBeginEnvironment{quote}{\quotefont\small}
\XeTeXlinebreaklocale ``zh''
\XeTeXlinebreakskip = 0pt plus 1pt
\linespread{1.0}
\hypersetup{
  colorlinks=true,
  linkcolor=[rgb]{0,0.37,0.53},
  citecolor=[rgb]{0,0.47,0.68},
  filecolor=[rgb]{0,0.37,0.53},
  urlcolor=[rgb]{0,0.37,0.53},
  pagebackref=true,
  linktoc=all,}
\renewcommand{\headrulewidth}{0.4pt}
\renewcommand{\footrulewidth}{0.4pt}
\pagestyle{fancy}
\fancyfoot[C]{} % Clear page number
\fancyhead[RE]{\headfootfont\small\leftmark} % 在偶数页的右侧显示章名
\fancyhead[LO]{\headfootfont\small\rightmark} % 在奇数页的左侧显示小节名
\fancyhead[LE,RO]{\headfootfont\small~\thepage~} % 在偶数页的左侧,奇数页的右侧显示页码
\usepackage{fontspec}
\setmainfont{Noto Serif CJK SC}
\usepackage{xeCJK}
\setCJKmainfont{WenQuanYi Micro Hei }
\setcounter{secnumdepth}{3}
\author{Bolun Liu}
\date{\today}
\title{第一章 财务成本基本原理}
\begin{document}

\maketitle
\setcounter{tocdepth}{2}
\tableofcontents

\section{企业的组织形式}
\label{sec:orgf77c801}
\begin{center}
\begin{tabular}{llll}
比较项目 & 个人独资企业 & 合伙企业 & 公司制企业\\
\hline
投资人承担的责任 & 无限债务责任 & 1. (1)普通合伙企业:无线、连带责任 & 有限债务责任\\
 &  & (2)有限合伙企业:无限+有限 & \\
 &  & (3)特殊普通合伙企业:1.故意或者重大过失人:无限+其他有限; & \\
 &  & 2. 非故意或重大过失,全体合伙人:无限连带责任 & \\
\hline
组建成本 & 低 & 居中 & 高\\
\end{tabular}
\end{center}
\section{财务管理的主要内容}
\label{sec:org2140f5c}
财务管理所谈的长期投资的特征投资的对象为\textbf{经营性长期资产} 。包括对\textbf{子公司}、\textbf{合营企业}、\textbf{联营企业}的长期股权投资。

\uline{净现值原则}
\section{财务管理的目标}
\label{sec:orgfcae984}
本书观点---\textbf{股东财富最大化目标}
\begin{center}
\begin{tabular}{ll}
财务管理的目标 & 未考虑因素\\
\hline
利润最大化 & 风险、利润取得的时间、投入资本与利润的关系\\
每股收益最大化 & 风险、利润取得的时间\\
股东财富最大化 & 股东投资资本不变,股价最大化 = 股东财富最大化\\
\end{tabular}
\end{center}
\begin{itemize}
\item 经营者的利益要求与协调
\begin{itemize}
\item 表现
\begin{itemize}
\item 道德风险
\item 逆向选择-措施
\end{itemize}
\item 措施
\begin{itemize}
\item 监督
\item 激励
\end{itemize}
\end{itemize}
\end{itemize}
\emph{监督成本}、\emph{激励成本}、\emph{偏离股东目标}的损失\_三者之和最小
\begin{itemize}
\item 债权人的利益相关要求
\begin{itemize}
\item 与股东冲突的表现
\begin{itemize}
\item 投资于比债权人预期\textbf{风险更高}的新项目
\item 发行新债
\end{itemize}
\item 协调办法
\begin{itemize}
\item 限制性条款
\item 如有损害利益意图,不再提供新的贷款或提前收回贷款。
\end{itemize}
\end{itemize}
\end{itemize}
\begin{center}
\begin{tabular}{ll}
类型 & 协调\\
\hline
合同利益相关者 & (1) \textbf{遵守合同}就可以\textbf{基本满足}合同利益相关者的要求\\
 & (2) \textbf{遵守到的规范的约束}\\
\hline
非合同利益相关者 & 享受的法律保护低于合同利益相关者,受公司的社会责任政策的影响很大\\
\end{tabular}
\end{center}
\begin{quote}
主张股东财富最大化并非忽略其他相关者利益。
股东权益是\textbf{剩余权益},只有在满足了其他方面的利益之后才会有股东的利益。公司必须交税、给员工发工资等,才能获得税后收益。
\end{quote}
\section{金融工具与金融市场}
\label{sec:org6ec551b}
固定收益证券:固定利率债券、浮动利率债券、\textbf{优先股}、\textbf{永续债}。
\textbf{可转换债券}属于\textbf{衍生证券}
\begin{itemize}
\item 有效资本市场的含义
\begin{center}
\begin{tabular}{ll}
市场有效的含义 & \textbf{价格}能够\textbf{同步地(迅速性)}、\textbf{完全地(完整性)}反映全部可用的\textbf{信息}\\
外部标志 & (1) \textbf{信息充分、均匀披露}: 信息能够充分披露和均匀分布,每个投资者在同一时间内得到\textbf{等量等质}的信息\\
 & (2) \textbf{价格迅速反映}:价格能迅速地根据有关信息变动,不会出现没有反应或反应迟钝\\
\end{tabular}
\end{center}
\item 资本市场有效的基础条件
\begin{itemize}
\item 理性的投资人
\item 独立的理性偏差
\item 套利
\end{itemize}
\end{itemize}
\begin{quote}
注:三个条件只要有一个存在,资本市场就会是有效的。
\end{quote}
\begin{itemize}
\item 有效资本市场对财务管理的意义
\begin{itemize}
\item 管理者\texttt{改变会计方法不会}提升企业的股票价值。
\item 金融投机不能使企业获利。
\item 关注自身股价对企业是有益的。
\end{itemize}
\item 资本市场有效性的检验
\end{itemize}
\begin{center}
\begin{tabular}{llll}
市场 & 检验方法 & 检验规则 & 理解\\
\hline
弱式有效资本市场 & 过滤检验 & 使用过滤规则交易的收益率,\textbf{不能持续超过}``简单购买/持有''策略的收益率 & 过滤原则无用\\
\hline
半强势有效资本市场 & 投资基金表现研究 & 投资基金的平均业绩,\textbf{不可能持续超过}是市场整体的收益率 & 基金经理人无用\\
\end{tabular}
\end{center}
\end{document}
